\documentclass[$if(acm-metadata.acmart-options)$$acm-metadata.acmart-options$$else$manuscript,screen$if(acm-metadata.final)$$else$,review$endif$$if(acm-metadata.anonymous)$,anonymous$endif$$endif$]{acmart}

$-- % Options for packages loaded elsewhere
$-- \PassOptionsToPackage{unicode$for(hyperrefoptions)$,$hyperrefoptions$$endfor$}{hyperref}
$-- \PassOptionsToPackage{hyphens}{url}
$-- $if(colorlinks)$
$-- \PassOptionsToPackage{dvipsnames,svgnames,x11names}{xcolor}
$-- $endif$
$-- $if(CJKmainfont)$
$-- \PassOptionsToPackage{space}{xeCJK}
$-- $endif$
$-- %
$-- $doc-class.tex()$

$-- \usepackage{amsmath,amssymb}
$-- $if(fontfamily)$
$-- \usepackage[$for(fontfamilyoptions)$$fontfamilyoptions$$sep$,$endfor$]{$fontfamily$}
$-- $else$
$-- \usepackage{lmodern}
$-- $endif$
$-- $if(linestretch)$
$-- \usepackage{setspace}
$-- $endif$
$-- \usepackage{iftex}
$-- \ifPDFTeX
$--   \usepackage[$if(fontenc)$$fontenc$$else$T1$endif$]{fontenc}
$--   \usepackage[utf8]{inputenc}
$--   \usepackage{textcomp} % provide euro and other symbols
$-- \else % if luatex or xetex
$-- $if(mathspec)$
$--   \ifXeTeX
$--     \usepackage{mathspec}
$--   \else
$--     \usepackage{unicode-math}
$--   \fi
$-- $else$
$--   \usepackage{unicode-math}
$-- $endif$
$--   \defaultfontfeatures{Scale=MatchLowercase}
$--   \defaultfontfeatures[\rmfamily]{Ligatures=TeX,Scale=1}
$-- $if(mainfont)$
$--   \setmainfont[$for(mainfontoptions)$$mainfontoptions$$sep$,$endfor$]{$mainfont$}
$-- $endif$
$-- $if(sansfont)$
$--   \setsansfont[$for(sansfontoptions)$$sansfontoptions$$sep$,$endfor$]{$sansfont$}
$-- $endif$
$-- $if(monofont)$
$--   \setmonofont[$for(monofontoptions)$$monofontoptions$$sep$,$endfor$]{$monofont$}
$-- $endif$
$-- $for(fontfamilies)$
$--   \newfontfamily{$fontfamilies.name$}[$for(fontfamilies.options)$$fontfamilies.options$$sep$,$endfor$]{$fontfamilies.font$}
$-- $endfor$
$-- $if(mathfont)$
$-- $if(mathspec)$
$--   \ifXeTeX
$--     \setmathfont(Digits,Latin,Greek)[$for(mathfontoptions)$$mathfontoptions$$sep$,$endfor$]{$mathfont$}
$--   \else
$--     \setmathfont[$for(mathfontoptions)$$mathfontoptions$$sep$,$endfor$]{$mathfont$}
$--   \fi
$-- $else$
$--   \setmathfont[$for(mathfontoptions)$$mathfontoptions$$sep$,$endfor$]{$mathfont$}
$-- $endif$
$-- $endif$
$-- $if(CJKmainfont)$
$--   \ifXeTeX
$--     \usepackage{xeCJK}
$--     \setCJKmainfont[$for(CJKoptions)$$CJKoptions$$sep$,$endfor$]{$CJKmainfont$}
$--   \fi
$-- $endif$
$-- $if(luatexjapresetoptions)$
$--   \ifLuaTeX
$--     \usepackage[$for(luatexjapresetoptions)$$luatexjapresetoptions$$sep$,$endfor$]{luatexja-preset}
$--   \fi
$-- $endif$
$-- $if(CJKmainfont)$
$--   \ifLuaTeX
$--     \usepackage[$for(luatexjafontspecoptions)$$luatexjafontspecoptions$$sep$,$endfor$]{luatexja-fontspec}
$--     \setmainjfont[$for(CJKoptions)$$CJKoptions$$sep$,$endfor$]{$CJKmainfont$}
$--   \fi
$-- $endif$
$-- \fi
$-- $if(zero-width-non-joiner)$
$-- %% Support for zero-width non-joiner characters.
$-- \makeatletter
$-- \def\zerowidthnonjoiner{%
$--   % Prevent ligatures and adjust kerning, but still support hyphenating.
$--   \texorpdfstring{%
$--     \textormath{\nobreak\discretionary{-}{}{\kern.03em}%
$--       \ifvmode\else\nobreak\hskip\z@skip\fi}{}%
$--   }{}%
$-- }
$-- \makeatother
$-- \ifPDFTeX
$--   \DeclareUnicodeCharacter{200C}{\zerowidthnonjoiner}
$-- \else
$--   \catcode`^^^^200c=\active
$--   \protected\def ^^^^200c{\zerowidthnonjoiner}
$-- \fi
$-- %% End of ZWNJ support
$-- $endif$
$-- % Use upquote if available, for straight quotes in verbatim environments
\IfFileExists{upquote.sty}{\usepackage{upquote}}{}
\IfFileExists{microtype.sty}{% use microtype if available
  \usepackage[$for(microtypeoptions)$$microtypeoptions$$sep$,$endfor$]{microtype}
  \UseMicrotypeSet[protrusion]{basicmath} % disable protrusion for tt fonts
}{}
$if(indent)$
$else$
\makeatletter
\@ifundefined{KOMAClassName}{% if non-KOMA class
  \IfFileExists{parskip.sty}{%
    \usepackage{parskip}
  }{% else
    \setlength{\parindent}{0pt}
    \setlength{\parskip}{6pt plus 2pt minus 1pt}}
}{% if KOMA class
  \KOMAoptions{parskip=half}}
\makeatother
$endif$
$-- $if(verbatim-in-note)$
$-- \usepackage{fancyvrb}
$-- $endif$
$-- \usepackage{xcolor}
$-- $if(geometry)$
$-- $if(beamer)$
$-- \geometry{$for(geometry)$$geometry$$sep$,$endfor$}
$-- $else$
$-- \usepackage[$for(geometry)$$geometry$$sep$,$endfor$]{geometry}
$-- $endif$
$-- $endif$
$-- $if(beamer)$
$-- \newif\ifbibliography
$-- $endif$
$-- $if(listings)$
$-- \usepackage{listings}
$-- \newcommand{\passthrough}[1]{#1}
$-- \lstset{defaultdialect=[5.3]Lua}
$-- \lstset{defaultdialect=[x86masm]Assembler}
$-- $endif$
$-- $if(lhs)$
$-- \lstnewenvironment{code}{\lstset{language=Haskell,basicstyle=\small\ttfamily}}{}
$-- $endif$
$-- $if(links-as-notes)$
$-- % Make links footnotes instead of hotlinks:
$-- \DeclareRobustCommand{\href}[2]{#2\footnote{\url{#1}}}
$-- $endif$
$-- $if(strikeout)$
$-- $-- also used for underline
$-- \usepackage[normalem]{ulem}
$-- $endif$
$-- \setlength{\emergencystretch}{3em} % prevent overfull lines
$-- $if(numbersections)$
$-- \setcounter{secnumdepth}{$if(secnumdepth)$$secnumdepth$$else$5$endif$}
$-- $else$
$-- \setcounter{secnumdepth}{-\maxdimen} % remove section numbering
$-- $endif$
$-- $if(beamer)$
$-- $else$
$-- $if(block-headings)$
$-- % Make \paragraph and \subparagraph free-standing
$-- \ifx\paragraph\undefined\else
$--   \let\oldparagraph\paragraph
$--   \renewcommand{\paragraph}[1]{\oldparagraph{#1}\mbox{}}
$-- \fi
$-- \ifx\subparagraph\undefined\else
$--   \let\oldsubparagraph\subparagraph
$--   \renewcommand{\subparagraph}[1]{\oldsubparagraph{#1}\mbox{}}
$-- \fi
$-- $endif$
$-- $endif$
$-- $if(pagestyle)$
$-- \pagestyle{$pagestyle$}
$-- $endif$
$-- $pandoc.tex()$
$-- $if(lang)$
$-- \ifLuaTeX
$-- \usepackage[bidi=basic]{babel}
$-- \else
$-- \usepackage[bidi=default]{babel}
$-- \fi
$-- $if(babel-lang)$
$-- \babelprovide[main,import]{$babel-lang$}
$-- $endif$
$-- $for(babel-otherlangs)$
$-- \babelprovide[import]{$babel-otherlangs$}
$-- $endfor$
$-- % get rid of language-specific shorthands (see #6817):
$-- \let\LanguageShortHands\languageshorthands
$-- \def\languageshorthands#1{}
$-- $endif$
$-- \ifLuaTeX
$--   \usepackage{selnolig}  % disable illegal ligatures
$-- \fi
$-- $if(dir)$
$-- \ifPDFTeX
$--   \TeXXeTstate=1
$--   \newcommand{\RL}[1]{\beginR #1\endR}
$--   \newcommand{\LR}[1]{\beginL #1\endL}
$--   \newenvironment{RTL}{\beginR}{\endR}
$--   \newenvironment{LTR}{\beginL}{\endL}
$-- \fi
$-- $endif$
$-- $if(biblio-config)$
$-- $if(natbib)$
$-- \usepackage[$natbiboptions$]{natbib}
$-- \bibliographystyle{$if(biblio-style)$$biblio-style$$else$plainnat$endif$}
$-- $endif$
$-- $if(biblatex)$
$-- \usepackage[$if(biblio-style)$style=$biblio-style$,$endif$$for(biblatexoptions)$$biblatexoptions$$sep$,$endfor$]{biblatex}
$-- $for(bibliography)$
$-- \addbibresource{$bibliography$}
$-- $endfor$
$-- $endif$
$-- $endif$
$-- $if(nocite-ids)$
$-- \nocite{$for(nocite-ids)$$it$$sep$, $endfor$}
$-- $endif$
$-- $if(csquotes)$
$-- \usepackage{csquotes}
$-- $endif$
$-- \IfFileExists{bookmark.sty}{\usepackage{bookmark}}{\usepackage{hyperref}}
$-- \IfFileExists{xurl.sty}{\usepackage{xurl}}{} % add URL line breaks if available
$-- \urlstyle{same} % disable monospaced font for URLs
$-- $if(verbatim-in-note)$
$-- \VerbatimFootnotes % allow verbatim text in footnotes
$-- $endif$
$-- \hypersetup{
$-- $if(title-meta)$
$--   pdftitle={$title-meta$},
$-- $endif$
$-- $if(author-meta)$
$--   pdfauthor={$author-meta$},
$-- $endif$
$-- $if(lang)$
$--   pdflang={$lang$},
$-- $endif$
$-- $if(subject)$
$--   pdfsubject={$subject$},
$-- $endif$
$-- $if(keywords)$
$--   pdfkeywords={$for(keywords)$$keywords$$sep$, $endfor$},
$-- $endif$
$-- $if(colorlinks)$
$--   colorlinks=true,
$--   linkcolor={$if(linkcolor)$$linkcolor$$else$Maroon$endif$},
$--   filecolor={$if(filecolor)$$filecolor$$else$Maroon$endif$},
$--   citecolor={$if(citecolor)$$citecolor$$else$Blue$endif$},
$--   urlcolor={$if(urlcolor)$$urlcolor$$else$Blue$endif$},
$-- $else$
$--   hidelinks,
$-- $endif$
$--   pdfcreator={LaTeX via pandoc}}
$-- $title.tex()$
$-- \begin{document}
$-- $before-body.tex()$
$-- $for(include-before)$
$-- $include-before$
$-- $endfor$
$-- $toc.tex()$
$-- $if(linestretch)$
$-- \setstretch{$linestretch$}
$-- $endif$
$-- $if(has-frontmatter)$
$-- \mainmatter
$-- $endif$
$-- $body$
$-- $before-bib.tex()$
$-- $if(has-frontmatter)$
$-- \backmatter
$-- $endif$
$-- $biblio.tex()$
$-- $for(include-after)$
$-- $include-after$
$-- $endfor$
$-- $after-body.tex()$
$-- \end{document}

%%
%% This is file `sample-manuscript.tex',
%% generated with the docstrip utility.
%%
%% The original source files were:
%%
%% samples.dtx  (with options: `manuscript')
%% 
%% IMPORTANT NOTICE:
%% 
%% For the copyright see the source file.
%% 
%% Any modified versions of this file must be renamed
%% with new filenames distinct from sample-manuscript.tex.
%% 
%% For distribution of the original source see the terms
%% for copying and modification in the file samples.dtx.
%% 
%% This generated file may be distributed as long as the
%% original source files, as listed above, are part of the
%% same distribution. (The sources need not necessarily be
%% in the same archive or directory.)
%%
%%
%% Commands for TeXCount
%TC:macro \cite [option:text,text]
%TC:macro \citep [option:text,text]
%TC:macro \citet [option:text,text]
%TC:envir table 0 1
%TC:envir table* 0 1
%TC:envir tabular [ignore] word
%TC:envir displaymath 0 word
%TC:envir math 0 word
%TC:envir comment 0 0
%%
%%
%% The first command in your LaTeX source must be the \documentclass command.


% Options for packages loaded elsewhere
\PassOptionsToPackage{unicode$for(hyperrefoptions)$,$hyperrefoptions$$endfor$}{hyperref}
\PassOptionsToPackage{hyphens}{url}
$if(colorlinks)$
\PassOptionsToPackage{dvipsnames,svgnames,x11names}{xcolor}
$endif$
$if(CJKmainfont)$
\PassOptionsToPackage{space}{xeCJK}
$endif$

\IfFileExists{bookmark.sty}{\usepackage{bookmark}}{\usepackage{hyperref}}

%% PANDOC PREAMBLE BEGINS
$pandoc.tex()$
%% PANDOC PREAMBLE ENDS

\setlength{\parindent}{10pt}
\setlength{\parskip}{0pt}

\hypersetup{
$if(title-meta)$
  pdftitle={$title-meta$},
$endif$
$if(author-meta)$
  pdfauthor={$author-meta$},
$endif$
$if(lang)$
  pdflang={$lang$},
$endif$
$if(subject)$
  pdfsubject={$subject$},
$endif$
$if(keywords)$
  pdfkeywords={$for(keywords)$$keywords$$sep$, $endfor$},
$endif$
$if(colorlinks)$
  colorlinks=true,
  linkcolor={$if(linkcolor)$$linkcolor$$else$Maroon$endif$},
  filecolor={$if(filecolor)$$filecolor$$else$Maroon$endif$},
  citecolor={$if(citecolor)$$citecolor$$else$Blue$endif$},
  urlcolor={$if(urlcolor)$$urlcolor$$else$Blue$endif$},
$else$
  hidelinks,
$endif$
  pdfcreator={LaTeX via pandoc, via quarto}}

%% \BibTeX command to typeset BibTeX logo in the docs
\AtBeginDocument{%
  \providecommand\BibTeX{{%
    Bib\TeX}}}

$_acmart_preamble.tex()$

%% end of the preamble, start of the body of the document source.
\begin{document}

$title.tex()$
$_acmart_authors.tex()$

%%  
%% The abstract is a short summary of the work to be presented in the
%% article.
$_acmart_abstract.tex()$

%%
%% The code below is generated by the tool at http://dl.acm.org/ccs.cfm.
%% Please copy and paste the code instead of the example below.
%%
$if(acm-metadata.ccs)$
$acm-metadata.ccs$
$endif$

%%
%% Keywords. The author(s) should pick words that accurately describe
%% the work being presented. Separate the keywords with commas.
$if(acm-metadata.keywords)$
\keywords{$for(acm-metadata.keywords)$$it$$sep$, $endfor$}
$endif$

$if(acm-metadata.teaser)$
\begin{verbatim}
  \begin{teaserfigure}
    \includegraphics[width=\textwidth]{$acm-metadata.teaser.image$}
    \caption{$acm-metadata.teaser.caption$}
    \Description{$acm-metadata.teaser.description$}
  \end{teaserfigure}
\end{verbatim}
$endif$

%%
%% This command processes the author and affiliation and title
%% information and builds the first part of the formatted document.
\maketitle

\setlength{\parskip}{-0.1pt}

$body$

%% begin pandoc before-bib
$-- $before-bib.tex()$
%% end pandoc before-bib
$-- $if(has-frontmatter)$
$-- \backmatter
$-- $endif$
%% begin pandoc biblio
$-- $biblio.tex()$
%% end pandoc biblio
%% begin pandoc include-after
$-- $for(include-after)$
$-- $include-after$
$-- $endfor$
%% end pandoc include-after
%% begin pandoc after-body
$-- $after-body.tex()$
%% end pandoc after-body

\end{document}
\endinput
%%
%% End of file `sample-manuscript.tex'.
